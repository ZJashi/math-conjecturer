%------------------------------------------------------------------------------%
% MATHEMATICAL HEADER v0.5.25.03.06
% BY VICTOR SEIXAS SOUZA
%------------------------------------------------------------------------------%

\usepackage[utf8]{inputenc}
\usepackage[T1]{fontenc}
\usepackage[english]{babel}
\usepackage[babel]{microtype}
\usepackage[a4paper, margin=1in]{geometry}
\usepackage[dvipsnames]{xcolor}

\usepackage{csquotes}

%% General AMS packages
\usepackage{amsmath,amssymb,amsthm}%,amsrefs}
%\usepackage{amsfonts}

\usepackage{mathrsfs}

%% For DeclarePairedDelimiterXPP and other helpers
\usepackage{mathtools}
\usepackage{commath}

\usepackage{thmtools}
\usepackage{thm-restate}

%% bbm font
\usepackage{bbm}
\usepackage{dsfont}
\usepackage{lmodern}
% use cmex rather than lmex
\usepackage{fixcmex}

\usepackage{cases}
\usepackage{enumitem}
%% Default Enumeration Style
\setlist[enumerate,1]{label=(\roman*)}


\usepackage{centernot}
\usepackage{booktabs}
\usepackage{multirow}
\usepackage{comment}
\usepackage{float}
\usepackage{scalerel}

%% Hyperlinks in the document
\usepackage[pdfborderstyle={/S/U/W 0},unicode]{hyperref}
\hypersetup{
    colorlinks=true,
    linkcolor=blue!85!black,
    citecolor=blue!85!black,
    urlcolor=blue!85!black,
}

%% Always use \Cref or \cref instead of \ref for references
\usepackage{cleveref}

%% For \ifblank command, so \norm{} automatically gives ||.||

\usepackage{etoolbox}

\usepackage{subfiles}
\usepackage{pdfpages}


%------------------------------------------------------------------------------%

%% Miscelaneous Options


%% Equation Numbering
\numberwithin{equation}{section}

%% Spacing between lines
\linespread{1.07}

%% Can also use \AtBeginEnvironment{bmatrix}{\setlength{\arraycolsep}{50pt}}
%% From etoolbox for instance, to change arraycolsep for an in given environment
\arraycolsep 1pt

%% arXiv citing
%\renewcommand{\eprint}[1]{\href{https://arxiv.org/abs/#1}{arXiv:#1}}
%\renewcommand{\eprint}[1]{arXiv:#1}

%%%%%%%%%%%%%%%%%%%%%%%%%%%%%%%%%%%%%%%%%%%%%%%%%%%%%%%%%%%%%%
%%%%%%%%%%%%%%%%%%%%%%%%%%%%%%%%%%%%%%%%%%%%%%%%%%%%%%%%%%%%%%
%%%%%%%%%%%%%%%%%%%%%%%%%%%%%%%%%%%%%%%%%%%%%%%%%%%%%%%%%%%%%%
%%%%%%%%%%%%%%%%%%%%%%%%%%%%%%%%%%%%%%%%%%%%%%%%%%%%%%%%%%%%%%
%%%%%%%%%%%%%%%            FIX THIS
%%%%%%%%%%%%%%%%%%%%%%%%%%%%%%%%%%%%%%%%%%%%%%%%%%%%%%%%%%%%%%
%%%%%%%%%%%%%%%%%%%%%%%%%%%%%%%%%%%%%%%%%%%%%%%%%%%%%%%%%%%%%%
%%%%%%%%%%%%%%%%%%%%%%%%%%%%%%%%%%%%%%%%%%%%%%%%%%%%%%%%%%%%%%

%% Make refcheck also check Cref and cref references
\ifdefined\checkunusedreferences
\usepackage{refcheck}

\makeatletter
\newcommand{\alsocheck}[1]{%
  \expandafter\let\csname @@\string#1\endcsname#1%
  \expandafter\DeclareRobustCommand\csname relax\string#1\endcsname[1]{%
    \csname @@\string#1\endcsname{##1}\@for\@temp:=##1\do{\wrtusdrf{\@temp}\wrtusdrf{{\@temp}}}}%
  \expandafter\let\expandafter#1\csname relax\string#1\endcsname
}
\newcommand{\alsocheckrange}[1]{%
  \expandafter\let\csname @@\string#1\endcsname#1%
  \expandafter\DeclareRobustCommand\csname relax\string#1\endcsname[2]{%
    \csname @@\string#1\endcsname{##1}{##2}\wrtusdrf{##1}\wrtusdrf{{##1}}\wrtusdrf{##2}\wrtusdrf{{##2}}}%
  \expandafter\let\expandafter#1\csname relax\string#1\endcsname
}
\makeatother

\alsocheck{\cref}
\alsocheck{\Cref}
\alsocheckrange{\crefrange}
\alsocheckrange{\Crefrange}

\else
\fi




%------------------------------------------------------------------------------%

%% Theorem Types


% Main style, for theorems
% \ifdefined\thmcolor
% \declaretheoremstyle{plain}
% \else
% \fi

% Non-italic style, for definitions and remarks
\ifdefined\defcolor
\declaretheoremstyle[
  headfont=\normalfont\bfseries,
  bodyfont=\normalfont,
  shaded={bgcolor=\defcolor}
]{noital}
\else
\declaretheoremstyle[
  headfont=\normalfont\bfseries,
  bodyfont=\normalfont,
]{noital}
\fi

%%% Theorem-style

% Numbered Versions
\declaretheorem[style=plain,numberwithin=section,name=Theorem]{theorem}
\declaretheorem[style=plain,sibling=theorem,name=Theorems]{theorems}
\declaretheorem[style=plain,sibling=theorem,name=Proposition]{proposition}
\declaretheorem[style=plain,sibling=theorem,name=Propositions]{propositions}
\declaretheorem[style=plain,sibling=theorem,name=Lemma]{lemma}
\declaretheorem[style=plain,sibling=theorem,name=Lemmas]{lemmas}
\declaretheorem[style=plain,sibling=theorem,name=Corollary]{corollary}
\declaretheorem[style=plain,sibling=theorem,name=Corollaries]{corollaries}
\declaretheorem[style=plain,sibling=theorem,name=Conjecture]{conjecture}
\declaretheorem[style=plain,sibling=theorem,name=Conjectures]{conjectures}
\declaretheorem[style=plain,sibling=theorem,name=Question]{question}
\declaretheorem[style=plain,sibling=theorem,name=Questions]{questions}
\declaretheorem[style=plain,sibling=theorem,name=Claim]{claim}
\declaretheorem[style=plain,sibling=theorem,name=Claims]{claims}
\declaretheorem[style=plain,sibling=theorem,name=Fact]{fact}
\declaretheorem[style=plain,sibling=theorem,name=Facts]{facts}
\declaretheorem[style=plain,sibling=theorem,name=Problem]{problem}
\declaretheorem[style=plain,sibling=theorem,name=Problems]{problems}
\declaretheorem[style=plain,sibling=theorem,name=Open Problem]{openproblem}
\declaretheorem[style=plain,sibling=theorem,name=Open Problems]{openproblems}
\declaretheorem[style=plain,sibling=theorem,name=Challenge]{challenge}
\declaretheorem[style=plain,sibling=theorem,name=Challenges]{challenges}
\declaretheorem[style=plain,sibling=theorem,name=Exercise]{exercise}
\declaretheorem[style=plain,sibling=theorem,name=Exercises]{exercises}
\declaretheorem[style=plain,sibling=theorem,name=Property]{property}
\declaretheorem[style=plain,sibling=theorem,name=Properties]{properties}
% Non Numbered Versions
\declaretheorem[style=plain,numbered=no,name=Theorem]{theoremn-n}
\declaretheorem[style=plain,numbered=no,name=Theorems]{theorems-n}
\declaretheorem[style=plain,numbered=no,name=Proposition]{proposition-n}
\declaretheorem[style=plain,numbered=no,name=Propositions]{propositions-n}
\declaretheorem[style=plain,numbered=no,name=Lemma]{lemma-n}
\declaretheorem[style=plain,numbered=no,name=Lemmas]{lemmas-n}
\declaretheorem[style=plain,numbered=no,name=Corollary]{corollary-n}
\declaretheorem[style=plain,numbered=no,name=Corollaries]{corollaries-n}
\declaretheorem[style=plain,numbered=no,name=Conjecture]{conjecture-n}
\declaretheorem[style=plain,numbered=no,name=Conjectures]{conjectures-n}
\declaretheorem[style=plain,numbered=no,name=Question]{question-n}
\declaretheorem[style=plain,numbered=no,name=Questions]{questions-n}
\declaretheorem[style=plain,numbered=no,name=Claim]{claim-n}
\declaretheorem[style=plain,numbered=no,name=Claims]{claims-n}
\declaretheorem[style=plain,numbered=no,name=Fact]{fact-n}
\declaretheorem[style=plain,numbered=no,name=Facts]{facts-n}
\declaretheorem[style=plain,numbered=no,name=Problem]{problem-n}
\declaretheorem[style=plain,numbered=no,name=Problems]{problems-n}
\declaretheorem[style=plain,numbered=no,name=Open Problem]{openproblem-n}
\declaretheorem[style=plain,numbered=no,name=Open Problems]{openproblems-n}
\declaretheorem[style=plain,numbered=no,name=Challenge]{challenge-n}
\declaretheorem[style=plain,numbered=no,name=Challenges]{challenges-n}
\declaretheorem[style=plain,numbered=no,name=Exercise]{exercise-n}
\declaretheorem[style=plain,numbered=no,name=Exercises]{exercises-n}
\declaretheorem[style=plain,numbered=no,name=Property]{property-n}
\declaretheorem[style=plain,numbered=no,name=Properties]{properties-n}

%%% Definition-style

% Numbered Versions
\declaretheorem[style=noital,sibling=theorem,name=Remark]{remark}
\declaretheorem[style=noital,sibling=theorem,name=Remarks]{remarks}
\declaretheorem[style=noital,sibling=theorem,name=Definition]{definition}
\declaretheorem[style=noital,sibling=theorem,name=Definitions]{definitions}
\declaretheorem[style=noital,sibling=theorem,name=Construction]{construction}
\declaretheorem[style=noital,sibling=theorem,name=Constructions]{constructions}
\declaretheorem[style=noital,sibling=theorem,name=Observation]{observation}
\declaretheorem[style=noital,sibling=theorem,name=Observations]{observations}
\declaretheorem[style=noital,sibling=theorem,name=Example]{example}
\declaretheorem[style=noital,sibling=theorem,name=Examples]{examples}
% Non Numbered Versions
\declaretheorem[style=noital,numbered=no,name=Remark]{remark-n}
\declaretheorem[style=noital,numbered=no,name=Remarks]{remarks-n}
\declaretheorem[style=noital,numbered=no,name=Definition]{definition-n}
\declaretheorem[style=noital,numbered=no,name=Definitions]{definitions-n}
\declaretheorem[style=noital,numbered=no,name=Construction]{construction-n}
\declaretheorem[style=noital,numbered=no,name=Constructions]{constructions-n}
\declaretheorem[style=noital,numbered=no,name=Observation]{observation-n}
\declaretheorem[style=noital,numbered=no,name=Observations]{observations-n}
\declaretheorem[style=noital,numbered=no,name=Example]{example-n}
\declaretheorem[style=noital,numbered=no,name=Examples]{examples-n}





%%%% REDEFINE A BUNCH OF EQUATIONS SO THEY WORK WELL WITH CREF

%% Create command for (1 + o(1))


%% Create command for intervals

%% Create a document that shows off each decision


%------------------------------------------------------------------------------%

%% General Semantics


%% Inline Definition
\newcommand{\indef}[1]{\emph{#1}}

%% Definied :=
\newcommand{\defined}{\mathrel{\coloneqq}}
%% Defines =:
\newcommand{\defines}{\mathrel{\eqqcolon}}

%% Parenthesis
\DeclarePairedDelimiter{\p}{\lparen}{\rparen}

%% Slanted
\renewcommand{\le}{\leqslant}
\renewcommand{\leq}{\leqslant}
\renewcommand{\ge}{\geqslant}
\renewcommand{\geq}{\geqslant}


%------------------------------------------------------------------------------%

%% Foundations


%% Quantifiers
\let\oldexists\exists
\let\exists\relax
\DeclareMathOperator{\exists}{\:\!\oldexists}
\let\oldforall\forall
\let\forall\relax
\DeclareMathOperator{\forall}{\:\!\oldforall}

%% Such that
\newcommand{\st}{\mathbin{\colon}}
%\newcommand{\st}{\colon}
%% Set
\undef{\set}
\DeclarePairedDelimiter{\set}{\lbrace}{\rbrace}

%% Empty Set
%\let\oldemptyset\emptyset
\undef{\emptyset}
\newcommand{\emptyset}{\varnothing}

%% Cardinality
\DeclarePairedDelimiter{\card}{\lvert}{\rvert}

%% Powerset
%% REDEFINE
\newcommand{\powerset}[1]{\mathcal{P}(#1)}

%% Union
\newcommand{\union}{\mathbin{\cup}}
%% Intersection
\newcommand{\inter}{\mathbin{\cap}}
%% Complement
\newcommand{\setcomp}[1]{{#1}^{\mathsf{c}}}
\newcommand{\comp}{\mathsf{c}}

%% Indicator function
\DeclareMathOperator{\ind}{\mathbf{1}}

%% Function from. eg: $f \from X \to Y$ for "f : X -> Y"
\newcommand{\from}{\colon}

%% Composition of functions
\DeclareMathOperator{\compo}{\circ}

%% Homomorphism
\DeclareMathOperator{\Hom}{Hom}
%\DeclareMathOperator{\hom}{hom}
%% Injection
\DeclareMathOperator{\Inj}{Inj}
\DeclareMathOperator{\inj}{inj}
%% Automorphism
\DeclareMathOperator{\Aut}{Aut}
\DeclareMathOperator{\aut}{aut}

%% Small sum
\newcommand{\ssum}{{\textstyle \sum}}
%% Small integral
\newcommand{\sint}{{\textstyle \int}}
%% Small product
\newcommand{\sprod}{{\textstyle \prod}}
%% Small cup
\newcommand{\scup}{{\textstyle \bigcup}}

%% Floor
\DeclarePairedDelimiter{\floor}{\lfloor}{\rfloor}
%% Ceil
\DeclarePairedDelimiter{\ceil}{\lceil}{\rceil}

%% Expectation Operator. eg $\expecs_{x \in G} f(x)$
\DeclareMathOperator*{\expecs}{\scalerel*{\mathbb{E}}{A_B}}

%------------------------------------------------------------------------------%

%% Calculus

%% Integral d
\renewcommand{\d}{\mathop{}\!\mathrm{d}}
%% Common integral variables
\newcommand{\dt}{\d t}
\newcommand{\ds}{\d s}
\newcommand{\du}{\d u}
\newcommand{\dx}{\d x}
\newcommand{\dy}{\d y}
\newcommand{\dz}{\d z}
\newcommand{\dw}{\d w}

%% Gradient
\newcommand{\grad}{\nabla}
\newcommand{\hess}{\nabla^2}

%------------------------------------------------------------------------------%

%% Number Theory

%% Loglogs
\DeclareMathOperator{\loglog}{log\,log}
\DeclareMathOperator{\logloglog}{log\,log\,log}
\DeclareMathOperator{\loglogloglog}{log\,log\,log\,log}

%% Divisibility
\newcommand{\divides}{\mathrel{\mid}}
\newcommand{\ndivides}{\mathrel{\nmid}}
%% Equivalent mod
\DeclareMathOperator{\eqmod}{\equiv}
%% Mod
\undef{\mod}
\newcommand{\mod}[1]{\ (\mathrm{mod}\ #1)}

%% Radical
\DeclareMathOperator{\rad}{rad}
\DeclareMathOperator{\lcm}{lcm}


%------------------------------------------------------------------------------%

%% Graph Theory


%% Extremal number in Graph Theory
\DeclareMathOperator{\ex}{ex}

%% Cayley graph
\DeclareMathOperator{\Cayley}{Cay}

%------------------------------------------------------------------------------%

%% Spaces


%% Absolute value
\undef{\abs}
\DeclarePairedDelimiterX{\abs}[1]
  {\lvert}{\rvert}{\ifblank{#1}{\,\cdot\,}{#1}}
%% Norm
\undef{\norm}
\DeclarePairedDelimiterX{\norm}[1]
  {\lVert}{\rVert}{\ifblank{#1}{\,\cdot\,}{#1}}
%% Inner Product
\DeclarePairedDelimiterX{\inner}[2]
  {\langle}{\rangle}{\ifblank{#1}{\,\cdot\,}{#1},\ifblank{#2}{\,\cdot\,}{#2}}


%% Complex conjugation
\newcommand*{\conj}[1]{\overline{#1}}
%% Convolution
\DeclareMathOperator{\conv}{\ast}

%% Real and Complex parts
\renewcommand{\Re}{\operatorname{Re}}
\renewcommand{\Im}{\operatorname{Im}}
\newcommand{\Arg}{\operatorname{Arg}}

%% Diameter
\DeclareMathOperator{\diam}{diam}
%% Support
\DeclareMathOperator{\supp}{supp}

%% Rank
\DeclareMathOperator{\rank}{rank}
%% Column
\DeclareMathOperator{\col}{col}
%% Row
\DeclareMathOperator{\row}{row}
%% Span (as vector space)
\DeclareMathOperator{\vspan}{span}
%% Projection
\DeclareMathOperator{\proj}{proj}

%% Lipschitz
\DeclareMathOperator{\Lip}{Lip}

%% Embedding hook
\DeclareMathOperator{\embeds}{\hookrightarrow}

%------------------------------------------------------------------------------%
% Groups

%% Order of element
\DeclareMathOperator{\ord}{ord}

%------------------------------------------------------------------------------%
% Matrices

%% Matrix
\DeclareMathOperator{\Mat}{Mat}
%% Transpose
\newcommand{\tran}{\top}
%% Trace
\DeclareMathOperator{\trace}{tr}
%% Determinant
%\DeclareMathOperator{\det}{det}
%% Singular values
\DeclareMathOperator{\smin}{s_{\min}}
\DeclareMathOperator{\smax}{s_{\max}}
%% Tensor
\DeclareMathOperator{\tensor}{\otimes}

%% Positive definite
\DeclareMathOperator{\pdleq}{\preccurlyeq}
\DeclareMathOperator{\pdgeq}{\succcurlyeq}
\DeclareMathOperator{\pdle}{\prec}
\DeclareMathOperator{\pdge}{\succ}

%% Hadamard product
\DeclareMathOperator{\hadprod}{\circ}

%------------------------------------------------------------------------------%

%% Measure and Probability


%% Distributed as
\DeclareMathOperator{\dist}{\sim}

%% General Measures
\DeclareMathOperator{\Unif}{Unif}
\DeclareMathOperator{\Haar}{Haar}
\DeclareMathOperator{\Leb}{Leb}
\DeclareMathOperator{\vol}{vol}

%% Discrete Measures
\DeclareMathOperator{\Ber}{Ber}
\DeclareMathOperator{\Bin}{Bin}
\DeclareMathOperator{\Rad}{Rad}
\DeclareMathOperator{\Geo}{Geo}
\DeclareMathOperator{\NegBin}{NB}
\DeclareMathOperator{\HyperGeo}{HypGeo}
\DeclareMathOperator{\Po}{Po}

%% Continuous Measures
\DeclareMathOperator{\Nor}{\mathcal{N}}
\DeclareMathOperator{\Gauss}{\mathcal{N}}
\DeclareMathOperator{\Exp}{Exp}
%\DeclareMathOperator{\GammaLaw}{\Gamma}
%\DeclareMathOperator{\BetaLaw}{\beta}
\DeclareMathOperator{\Weibull}{W}
\DeclareMathOperator{\Student}{S}

%% Essensial Supremum
\DeclareMathOperator*{\esssup}{ess\,sup}

%% Total Variation
\DeclareMathOperator{\TV}{TV}
%% Total Variation Distance
\DeclareMathOperator{\dtv}{d_{\TV}}

%% Convergence
\newcommand{\converges}[1]{\xrightarrow[]{#1}}
\newcommand{\convlaw}{\converges{\operatorname{law}}}
\newcommand{\convprob}{\converges{\operatorname{prob}}}
\newcommand{\convdist}{\converges{\operatorname{dist}}}
\newcommand{\convmeas}{\converges{\operatorname{meas}}}
\newcommand{\convtv}{\converges{\TV}}
\newcommand{\convas}{\converges{\operatorname{a.s.}}}

%% With high probability
\newcommand{\whp}{w.h.p. }
%% Independently identically distributed
\newcommand{\iid}{i.i.d. }
%% Almost surelly
\newcommand{\as}{a.s. }
% NOTE MAYBE LATEX TREAT . as an end of sentence

%% Probability and Expectations
%  Note that these commands created with DeclarePairedDelimiter do not
%  work with line breaks. If you need to span a probability over several lines,
%  use \prob instead of \Prob
%% Conditional Symbol
\DeclareMathDelimiter{\given}
  {\mathbin}{symbols}{"6A}{largesymbols}{"0C}
%% Probability
\DeclareMathOperator{\Prob}{\mathbb{P}}
\DeclarePairedDelimiterXPP{\prob}[1]
  {\Prob}{\lparen}{\rparen}{}
  {\renewcommand{\given}{\nonscript\;\delimsize\vert\nonscript\;\mathopen{}}#1}
%% Expectation
\DeclareMathOperator{\Expec}{\mathbb{E}}
\DeclarePairedDelimiterXPP{\expec}[1]
  {\Expec}{\lparen}{\rparen}{}
  {\renewcommand{\given}{\nonscript\;\delimsize\vert\nonscript\;\mathopen{}}#1}
%% Variance
\DeclareMathOperator{\Var}{Var}
\DeclarePairedDelimiterXPP{\var}[1]
  {\Var}{\lparen}{\rparen}{}
  {\renewcommand{\given}{\nonscript\;\delimsize\vert\nonscript\;\mathopen{}}#1}
%% Covariance (Binary only, for more than two variables, use \cov)
\DeclareMathOperator{\Cov}{Cov}
\DeclarePairedDelimiterXPP{\cov}[2]
  {\Cov}{\lparen}{\rparen}{}{#1,#2}

%% TODO
%% Make a command to make new measures
%% Something like
%% \probabilities{\mathbb{P}}{\prob}{\mathbb{E}}{\expec}
%% https://tex.stackexchange.com/questions/65780/macro-defining-macro/65781#65781
\DeclareMathOperator*{\argmax}{arg\,max}
\DeclareMathOperator*{\argmin}{arg\,min}

%------------------------------------------------------------------------------%

%% Good practices to remember

%% Dots
% \dotsc for “dots with commas”
% \dotsb for “dots with binary operators/relations”
% \dotsm for “multiplication dots”
% \dotsi for “dots with integrals”
% \dotso for “other dots” (none of the above)

%% Never use \left \right and \middle
% Control the parenthesis sizes manually
% That means using \bigl \bigr and \bigm
% Or the corresponding \Bigr \biggr and \Biggr
% Avoid using just \big for spacing reasons

%------------------------------------------------------------------------------%

%% Font Shortcuts

%% Varepsilon is too common to have sucha a long name
\newcommand{\eps}{\varepsilon}

%% Blackboard font
\let\AA\relax
\let\LL\relax
\let\SS\relax
\newcommand{\AA}{\mathbb{A}}
\newcommand{\BB}{\mathbb{B}}
\newcommand{\CC}{\mathbb{C}}
\newcommand{\DD}{\mathbb{D}}
\newcommand{\EE}{\mathbb{E}}
\newcommand{\FF}{\mathbb{F}}
\newcommand{\GG}{\mathbb{G}}
\newcommand{\HH}{\mathbb{H}}
\newcommand{\II}{\mathbb{I}}
\newcommand{\JJ}{\mathbb{J}}
\newcommand{\KK}{\mathbb{K}}
\newcommand{\LL}{\mathbb{L}}
\newcommand{\MM}{\mathbb{M}}
\newcommand{\NN}{\mathbb{N}}
\newcommand{\OO}{\mathbb{O}}
\newcommand{\PP}{\mathbb{P}}
\newcommand{\QQ}{\mathbb{Q}}
\newcommand{\RR}{\mathbb{R}}
\newcommand{\SS}{\mathbb{S}}
\newcommand{\TT}{\mathbb{T}}
\newcommand{\UU}{\mathbb{U}}
\newcommand{\VV}{\mathbb{V}}
\newcommand{\WW}{\mathbb{W}}
\newcommand{\XX}{\mathbb{X}}
\newcommand{\YY}{\mathbb{Y}}
\newcommand{\ZZ}{\mathbb{Z}}

%% Caligraphic font
\newcommand{\cA}{\mathcal{A}}
\newcommand{\cB}{\mathcal{B}}
\newcommand{\cC}{\mathcal{C}}
\newcommand{\cD}{\mathcal{D}}
\newcommand{\cE}{\mathcal{E}}
\newcommand{\cF}{\mathcal{F}}
\newcommand{\cG}{\mathcal{G}}
\newcommand{\cH}{\mathcal{H}}
\newcommand{\cI}{\mathcal{I}}
\newcommand{\cJ}{\mathcal{J}}
\newcommand{\cK}{\mathcal{K}}
\newcommand{\cL}{\mathcal{L}}
\newcommand{\cM}{\mathcal{M}}
\newcommand{\cN}{\mathcal{N}}
\newcommand{\cO}{\mathcal{O}}
\newcommand{\cP}{\mathcal{P}}
\newcommand{\cQ}{\mathcal{Q}}
\newcommand{\cR}{\mathcal{R}}
\newcommand{\cS}{\mathcal{S}}
\newcommand{\cT}{\mathcal{T}}
\newcommand{\cU}{\mathcal{U}}
\newcommand{\cV}{\mathcal{V}}
\newcommand{\cW}{\mathcal{W}}
\newcommand{\cX}{\mathcal{X}}
\newcommand{\cY}{\mathcal{Y}}
\newcommand{\cZ}{\mathcal{Z}}

%% Roman font
\newcommand{\rA}{\mathrm{A}}
\newcommand{\rB}{\mathrm{B}}
\newcommand{\rC}{\mathrm{C}}
\newcommand{\rD}{\mathrm{D}}
\newcommand{\rE}{\mathrm{E}}
\newcommand{\rF}{\mathrm{F}}
\newcommand{\rG}{\mathrm{G}}
\newcommand{\rH}{\mathrm{H}}
\newcommand{\rI}{\mathrm{I}}
\newcommand{\rJ}{\mathrm{J}}
\newcommand{\rK}{\mathrm{K}}
\newcommand{\rL}{\mathrm{L}}
\newcommand{\rM}{\mathrm{M}}
\newcommand{\rN}{\mathrm{N}}
\newcommand{\rO}{\mathrm{O}}
\newcommand{\rP}{\mathrm{P}}
\newcommand{\rQ}{\mathrm{Q}}
\newcommand{\rR}{\mathrm{R}}
\newcommand{\rS}{\mathrm{S}}
\newcommand{\rT}{\mathrm{T}}
\newcommand{\rU}{\mathrm{U}}
\newcommand{\rV}{\mathrm{V}}
\newcommand{\rW}{\mathrm{W}}
\newcommand{\rX}{\mathrm{X}}
\newcommand{\rY}{\mathrm{Y}}
\newcommand{\rZ}{\mathrm{Z}}

%% Bold font
\newcommand{\bfA}{\mathbf{A}}
\newcommand{\bfB}{\mathbf{B}}
\newcommand{\bfC}{\mathbf{C}}
\newcommand{\bfD}{\mathbf{D}}
\newcommand{\bfE}{\mathbf{E}}
\newcommand{\bfF}{\mathbf{F}}
\newcommand{\bfG}{\mathbf{G}}
\newcommand{\bfH}{\mathbf{H}}
\newcommand{\bfI}{\mathbf{I}}
\newcommand{\bfJ}{\mathbf{J}}
\newcommand{\bfK}{\mathbf{K}}
\newcommand{\bfL}{\mathbf{L}}
\newcommand{\bfM}{\mathbf{M}}
\newcommand{\bfN}{\mathbf{N}}
\newcommand{\bfO}{\mathbf{O}}
\newcommand{\bfP}{\mathbf{P}}
\newcommand{\bfQ}{\mathbf{Q}}
\newcommand{\bfR}{\mathbf{R}}
\newcommand{\bfS}{\mathbf{S}}
\newcommand{\bfT}{\mathbf{T}}
\newcommand{\bfU}{\mathbf{U}}
\newcommand{\bfV}{\mathbf{V}}
\newcommand{\bfW}{\mathbf{W}}
\newcommand{\bfX}{\mathbf{X}}
\newcommand{\bfY}{\mathbf{Y}}
\newcommand{\bfZ}{\mathbf{Z}}

\newcommand{\bfa}{\mathbf{a}}
\newcommand{\bfb}{\mathbf{b}}
\newcommand{\bfc}{\mathbf{c}}
\newcommand{\bfd}{\mathbf{d}}
\newcommand{\bfe}{\mathbf{e}}
\newcommand{\bff}{\mathbf{f}}
\newcommand{\bfg}{\mathbf{g}}
\newcommand{\bfh}{\mathbf{h}}
\newcommand{\bfi}{\mathbf{i}}
\newcommand{\bfj}{\mathbf{j}}
\newcommand{\bfk}{\mathbf{k}}
\newcommand{\bfl}{\mathbf{l}}
\newcommand{\bfm}{\mathbf{m}}
\newcommand{\bfn}{\mathbf{n}}
\newcommand{\bfo}{\mathbf{o}}
\newcommand{\bfp}{\mathbf{p}}
\newcommand{\bfq}{\mathbf{q}}
\newcommand{\bfr}{\mathbf{r}}
\newcommand{\bfs}{\mathbf{s}}
\newcommand{\bft}{\mathbf{t}}
\newcommand{\bfu}{\mathbf{u}}
\newcommand{\bfv}{\mathbf{v}}
\newcommand{\bfw}{\mathbf{w}}
\newcommand{\bfx}{\mathbf{x}}
\newcommand{\bfy}{\mathbf{y}}
\newcommand{\bfz}{\mathbf{z}}
\newcommand{\bfell}{\boldsymbol{\ell}}

\newcommand{\bfGamma}{\mathbf{\Gamma}}
\newcommand{\bfDelta}{\mathbf{\Delta}}
\newcommand{\bfTheta}{\mathbf{\Theta}}
\newcommand{\bfUpsilon}{\mathbf{\Upsilon}}
\newcommand{\bfLambda}{\mathbf{\Lambda}}
\newcommand{\bfXi}{\mathbf{\Xi}}
\newcommand{\bfPi}{\mathbf{\Pi}}
\newcommand{\bfSigma}{\mathbf{\Sigma}}
\newcommand{\bfPhi}{\mathbf{\Phi}}
\newcommand{\bfPsi}{\mathbf{\Psi}}
\newcommand{\bfOmega}{\mathbf{\Omega}}

\newcommand{\bfalpha}{\boldsymbol{\alpha}}
\newcommand{\bfbeta}{\boldsymbol{\beta}}
\newcommand{\bfgamma}{\boldsymbol{\gamma}}
\newcommand{\bfdelta}{\boldsymbol{\delta}}
\newcommand{\bfeps}{\boldsymbol{\eps}}
\newcommand{\bfepsilon}{\boldsymbol{\epsilon}}
\newcommand{\bfvarepsilon}{\boldsymbol{\varepsilon}}
\newcommand{\bfzeta}{\boldsymbol{\zeta}}
\newcommand{\bfeta}{\boldsymbol{\eta}}
\newcommand{\bftheta}{\boldsymbol{\theta}}
\newcommand{\bfvartheta}{\boldsymbol{\vartheta}}
\newcommand{\bfupsilon}{\boldsymbol{\upsilon}}
\newcommand{\bfiota}{\boldsymbol{\iota}}
\newcommand{\bfkappa}{\boldsymbol{\kappa}}
\newcommand{\bflambda}{\boldsymbol{\lambda}}
\newcommand{\bfmu}{\boldsymbol{\mu}}
\newcommand{\bfnu}{\boldsymbol{\nu}}
\newcommand{\bfxi}{\boldsymbol{\xi}}
\newcommand{\bfpi}{\boldsymbol{\pi}}
\newcommand{\bfrho}{\boldsymbol{\rho}}
\newcommand{\bfvarrho}{\boldsymbol{\varrho}}
\newcommand{\bfsigma}{\boldsymbol{\sigma}}
\newcommand{\bftau}{\boldsymbol{\tau}}
\newcommand{\bfphi}{\boldsymbol{\phi}}
\newcommand{\bfvarphi}{\boldsymbol{\varphi}}
\newcommand{\bfchi}{\boldsymbol{\chi}}
\newcommand{\bfpsi}{\boldsymbol{\psi}}
\newcommand{\bfomega}{\boldsymbol{\omega}}

\newcommand{\bfzero}{\mathbf{0}}
\newcommand{\bfone}{\mathbf{1}}
%------------------------------------------------------------------------------%
