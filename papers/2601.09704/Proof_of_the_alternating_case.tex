\section{Proof of the alternating case}\label{sec: Proof of the alternating case}

\begin{thm}\label{thm: quotient simulate_alt}
Let $\epsilon>0$ be a real number. For all $n\ge 1$, let $A_n\in\Alt_n(\Z)$ be the same as in \Cref{thm: exponential convergence_alt}, and $a\ge 2$ be an integer. Then we have
$$D_{L^1}\left(\mathcal{L}(\Cok(A_{2n}/a)),\mu_{2\infty}^{\alt,(a)}\right)=O_{a,\epsilon}(\exp(-\epsilon\Omega_a(n))),$$
$$D_{L^1}\left(\mathcal{L}(\Cok(A_{2n+1}/a)),\mu_{2\infty+1}^{\alt,(a)}\right)=O_{a,\epsilon}(\exp(-\epsilon\Omega_a(n))).$$
Here, $\mu_{2\infty}^{\alt,(a)},\mu_{2\infty+1}^{\alt,(a)}$ are the same as in \Cref{prop: uniform model exponential convergence_alt}. 
\end{thm}

Before we return to the proof of \Cref{thm: quotient simulate_alt}, we see how it implies \Cref{thm: exponential convergence_alt}.

\begin{proof}[Proof of \Cref{thm: exponential convergence_alt}, assuming \Cref{thm: quotient simulate_alt}]
Following \Cref{prop: simplest matrix_alt}, when $G\notin\mathcal{S}_P$, it is clear that
$$\mathbf{P}(\Cok(A_{2n})_P\simeq G)=\mathbf{P}(\Cok_{\tors}(A_{2n+1})_P\simeq G)=0.$$
Therefore, for the rest of this proof, we assume $G\in\mathcal{S}_P$. Let $a:=p_1^{e_{p_1}}\cdots p_l^{e_{p_l}}$ be the same as in \Cref{cor: globally same pairing_alt}. In this case, for all $1\le i\le l$ and $n\ge 1$,
$$\mu_{2n}^{\alt,(p_i^{e_{p_i}})}\left(G_{p_i}\right)=\mathbf{P}\left(\Cok(A_{2n}^{(p_i')})\simeq G_{p_i}\right),$$
$$\mu_{2n+1}^{\alt,(p_i^{e_{p_i}})}\left(G_{p_i}\oplus(\Z/p_i^{e_{p_i}}\Z)\right)=\mathbf{P}\left(\Cok_{\tors}(A_{2n+1}^{(p_i')})\simeq G_{p_i}\right).$$
Here, $A_{2n}^{(p_i')}\in\Alt_{2n}(\Z_{p_i}),A_{2n+1}^{(p_i')}\in\Alt_{2n+1}(\Z_{p_i})$ are Haar-distributed. For the even size case, applying \cite[Theorem 3.9]{bhargava2015modeling}, we have
\begin{align}\label{eq: uniform model even size_alt}
\begin{split}
\mu_{2\infty}^{\alt,(a)}\left(G\right)&=\prod_{i=1}^l\mu_{2\infty}^{\alt,(p_i^{e_{p_i}})}\left(G_{p_i}\right)\\
&=\prod_{i=1}^l\lim_{n\rightarrow\infty}\mathbf{P}\left(\Cok(A_{2n}^{(p_i')})\simeq G_{p_i}\right)\\
&=\prod_{i=1}^l\frac{|G_{p_i}|}{|\Sp(G_{p_i})|}\prod_{k\ge 1}(1-p_i^{1-2k})\\
&=\frac{|G|}{|\Sp(G)|}\prod_{i=1}^l\prod_{k\ge 1}(1-p_i^{1-2k}).
\end{split}
\end{align}
For the odd size case, following \cite[Section 4]{nguyen2025local}, which gives a detailed exposition of \cite[Theorem 3.11]{bhargava2015modeling}, we have
\begin{align}\label{eq: uniform model odd size_alt}
\begin{split}
\mu_{2\infty+1}^{\alt,(a)}\left(G\oplus(\Z/a\Z)\right)&=\prod_{i=1}^l\mu_{2\infty+1}^{\alt,(p_i^{e_{p_i}})}\left(G_{p_i}\oplus(\Z/p_i^{e_{p_i}}\Z)\right)\\
&=\prod_{i=1}^l\lim_{n\rightarrow\infty}\mathbf{P}\left(\Cok_{\tors}(A_{2n+1}^{(p_i')})\simeq G_{p_i}\right)\\
&=\prod_{i=1}^l\frac{1}{|\Sp(G_{p_i})|}\prod_{k\ge 1}(1-p_i^{-1-2k})\\
&=\frac{1}{|\Sp(G)|}\prod_{i=1}^l\prod_{k\ge 1}(1-p_i^{-1-2k}).
\end{split}
\end{align}

Therefore, due to our assumption in \Cref{thm: quotient simulate_alt}, we have
$$\left|\mathbf{P}\left(\Cok(A_{2n})_P\simeq G\right)-\frac{|G|}{|\Sp(G)|}\prod_{i=1}^l\prod_{k\ge 1}(1-p_i^{1-2k})\right|=O_{a,\epsilon}(\exp(-\epsilon\Omega_a(n))),$$
$$\left|\mathbf{P}\left(\Cok(A_{2n+1})_P\simeq G\right)-\frac{1}{|\Sp(G)|}\prod_{i=1}^l\prod_{k\ge 1}(1-p_i^{-1-2k})\right|=O_{a,\epsilon}(\exp(-\epsilon\Omega_a(n))).$$
Notice that the integer $a$ defined in \Cref{cor: globally same pairing_alt} depends on $G,P$. Therefore, we can rewrite $O_{a,\epsilon}(\exp(-\epsilon\Omega_a(n)))$ as $O_{G,P,\epsilon}(\exp(-\epsilon\Omega_{G,P}(n)))$. This completes the proof.
\end{proof}

Now we return to our proof of \Cref{thm: quotient simulate_alt}. Let us start from the following proposition.

\begin{prop}\label{prop: universal transition_alt}
Let $\epsilon>0$ be a real number, and $a\ge 2$ be an integer with prime factorization $a=p_1^{e_{p_1}}\cdots p_l^{e_{p_l}}$. Suppose $A_n\in\Alt_n(\Z)$ is a fixed matrix that satisfies $\mathcal{E}_{n,p_i}^{\alt}$ for $1\le i\le l$. Let $\xi_1,\ldots,\xi_n$ be independent $\epsilon$-balanced random integers, and $\xi_1',\ldots,\xi_n'$ be independent and uniformly distributed in $\{0,1,\ldots,a-1\}$. Let $\bm{\xi}=(\xi_1,\ldots,\xi_n),\bm{\xi}'=(\xi_1',\ldots,\xi_n')\in\Z^n$. Then we have
$$
D_{L^2}\left(\Cok
\begin{pmatrix}A_n/a & \bm{\xi}/a\\
-(\bm{\xi}/a)^T & 0
\end{pmatrix}
,\Cok\begin{pmatrix}A_n/a & \bm{\xi}'/a\\
-(\bm{\xi}'/a)^T & 0
\end{pmatrix}\right)=O_{a,\epsilon}(\exp(-\epsilon\Omega_{a}(n))).\\
$$
\end{prop}

\begin{proof}
For all $1\le i\le l$, by \Cref{lem: full rank principal minor}, there exists $I_{p_i}\subset[n]$ with $\# I_{p_i}=n-\rank(A_n/{p_i})$, such that $(A_n/p_i)_{I_{p_i}^c\times I_{p_i}^c}$ has full rank. By \ref{item: large rank_alt}, we have $\#I_{p_i}\le n^{2/3}$. 

Now, we treat the $p
_i$-part of the cokernel with its quasi-pairing, where we reduce all matrix entries modulo $p_i^{e_{p_i}}$. By paired row-column operations, we can use the invertible matrix $(A_n/p_i^{e_{p_i}})_{I_{p_i}^c\times I_{p_i}^c}$ to eliminate the components of $\bm{\xi}/p_i^{e_{p_i}}$ and $(\bm{\xi}/p_i^{e_{p_i}})^T$ indexed from $I_{p_i}^c$, while the alternating structure is preserved. In this way, the new added column $\bm{\xi}/p_i^{e_{p_i}}$ will be transformed into the form
$$\bm{w}_i:=
(\bm{\xi}/p_i^{e_{p_i}})_{I_{p_i}}-(A_n/p_i^{e_{p_i}})_{I_{p_i}\times I_{p_i}^c}(A_n/p_i^{e_{p_i}})_{I_{p_i}^c\times I_{p_i}^c}^{-1}(\bm{\xi}/p_i^{e_{p_i}})_{I_{p_i}^c}\in(\Z/p_i^{e_{p_i}}\Z)^{\#I_{p_i}}.\\
$$
Here, we ignore and skip the components with index $I_{p_i}^c$ because they have been eliminated to zero. When $i$ runs over all integers from $1$ to $l$, we obtain a combination of vectors $(\bm{w}_i)_{1\le i\le l}$,
and this combination is a random variable in $\left((\Z/p_i^{e_{p_i}}\Z)^{\#I_{p_i}}\right)_{1\le i\le l}$. If we replace $\xi_1,\ldots,\xi_n$ by the input $\xi_1',\ldots,\xi_n'$, we can obtain a combination of vectors $(\bm{w}_i')_{1\le i\le l}\in\left((\Z/p_i^{e_{p_i}}\Z)^{\#I_{p_i}}\right)_{1\le i\le l}$ in the similar way. It is clear that $(\bm{w}_i')_{1\le i\le l}$ is uniformly distributed. We only need to prove that $$D_{L^2}\left((\bm{w}_i)_{1\le i\le l},(\bm{w}_i')_{1\le i\le l}\right)=
O_{a,\epsilon}(\exp(-\epsilon\Omega_a(n))).
$$
To achieve this, applying Parseval's identity, we have
$$D_{L^2}^2\left((\bm{w}_i)_{1\le i\le l},(\bm{w}_i')_{1\le i\le l}\right)=\prod_{i=1}^lp_i^{-e_{p_i}\#I_{p_i}}\sum\left|\E_{\bm{\xi},z}\exp\left(\sum_{i=1}^l\frac{2\pi \sqrt{-1}}{p_i^{e_{p_i}}}\bm{\alpha}_i\cdot\bm{w}_i\right)\right|^2.$$
Here, the sum on the right hand side ranges through the nonzero elements in the set
$$
\mathcal{V}:=\{(\bm{\alpha}_i)_{1\le i\le l}: \bm{\alpha}_i:=(\alpha_{i,j_i})_{j_i\in I_{p_i}}\in(\Z/p_i^{e_{p_i}}\Z)^{\#I_{p_i}}\}.
$$
Therefore, it suffices to show that for all nonzero $(\bm{\alpha}_i)_{1\le i\le l}\in\mathcal{V}$, we have
\begin{multline}\label{eq: small at nonzero vector_alt}
\left|\E_{\bm{\xi},z}\exp\left(\sum_{i=1}^l\frac{2\pi \sqrt{-1}}{p_i^{e_{p_i}}}\bm{\alpha}_i\cdot\left((\bm{\xi}/p_i^{e_{p_i}})_{I_{p_i}}-(A_n/p_i^{e_{p_i}})_{I_{p_i}\times I_{p_i}^c}(A_n/p_i^{e_{p_i}})_{I_{p_i}^c\times I_{p_i}^c}^{-1}(\bm{\xi}/p_i^{e_{p_i}})_{I_{p_i}^c}\right)\right)\right|\\=
O_{a,\epsilon}(\exp(-\epsilon\Omega_a(n))).
\end{multline}
Without loss of generality, we can let $\bm{\alpha}_1\ne 0$. In this case, let $k$ be the smallest valuation of all the entries of $\bm{\alpha}_1$. Then $0\le k<e_{p_1}$, and there exists 
$\bm{\beta}_1\in(\Z/p_1^{e_{p_1}}\Z)^{\#I_{p_1}}$, such that
$$\bm{\alpha}_1=p_1^k\bm{\beta_1},\bm{\beta}_1/p_1\ne 0.$$
Now, we consider the $n$-dimensional vector 
$$\left(-(\bm{\beta}_1/p_1),(A_n/p_1)_{I_{p_1}^c\times I_{p_1}^c}^{-1}(A_n/p_1)_{I_{p_1}^c\times I_{p_1}}(\bm{\beta}_1/p_1)\right)\in\F_p^n.$$ 
Here, we rearranged the entries: $-(\bm{\beta}_1/p_1)$ refers to the index set $I_{p_1}$, and $$(A_n/p_1)_{I_{p_1}^c\times I_{p_1}^c}^{-1}(A_n/p_1)_{I_{p_1}^c\times I_{p_1}}(\bm{\beta}_1/p_1)$$ 
refers to the index set $I_{p_1}^c$.
This vector is orthogonal to the rows of $A_n/p_1$ with index in $I_{p_1}^c$, which has cardinality $\ge n-n^{2/3}$. By \ref{item: ortho nonsparse_alt}, we deduce that the vector $$(A_n/p_1^{e_{p_1}})_{I_{p_1}^c\times I_{p_1}^c}^{-1}(A_n/p_1^{e_{p_1}})_{I_{p_1}^c\times I_{p_1}}\bm{\alpha}_1=p^k(A_n/p_1^{e_{p_1}})_{I_{p_1}^c\times I_{p_1}^c}^{-1}(A_n/p_1^{e_{p_1}})_{I_{p_1}^c\times I_{p_1}}\bm{\beta}_1$$ 
has at least $\frac{n}{100}-n^{2/3}$ nonzero entries. Applying \Cref{lem: expectation of not concentrated}, we have
$$\text{LHS}\eqref{eq: small at nonzero vector_alt}\le\exp(-\frac{\epsilon (n/100-n^{2/3})}{a^2})=O_{a,\epsilon}(\exp(-\epsilon\Omega_a(n))).$$
This completes the proof.
\end{proof}

\begin{cor}\label{cor: integral universal transition_alt}
Let $\epsilon>0$ be a real number, and $n\ge 1$. Let $P=\{p_1,\ldots,p_l\}$ be a finite set of primes, and $G^{(1)},G^{(2)}\in\mathcal{S}_p$. Let $\xi_1,\xi_2,\ldots$ be independent $\epsilon$-balanced random integers.
\begin{enumerate}
\item (From even to odd) Let $A_{2n}\in\Alt_{2n}(\Z)$ be a fixed matrix that satisfies $\mathcal{E}_{2n,p_i}^{\alt}$ for $1\le i\le l$, and $\Cok(A_{2n})_P\simeq G^{(1)}$. Then, we have
\begin{multline}\label{eq: even to odd isomorphic to G2_alt}
\mathbf{P}\left(\Cok_{\tors}\left(
\begin{array}{c|c}
A_{2n} & 
\begin{matrix}
\xi_1 \\
\vdots \\
\xi_{2n}
\end{matrix}
\\ \hline
\begin{matrix}
-\xi_1 & \cdots & -\xi_{2n}
\end{matrix}
& 0
\end{array}
\right)_P\simeq G^{(2)},\corank\left(
\begin{array}{c|c}
A_{2n} & 
\begin{matrix}
\xi_1 \\
\vdots \\
\xi_{2n}
\end{matrix}
\\ \hline
\begin{matrix}
-\xi_1 & \cdots & -\xi_{2n}
\end{matrix}
& 0
\end{array}
\right)=1\right)\\
=\prod_{i=1}^l\mathbf{P}\left(G_{p_i}^{(2)},\text{ odd }\bigg| G_{p_i}^{(1)},\text{ even}\right)+O_{G^{(1)},G^{(2)},P,\epsilon}(\exp(-\epsilon\Omega_{G^{(1)},G^{(2)},P}(n))).
\end{multline}
\item (From odd to even) Let $A_{2n+1}\in\Alt_{2n+1}(\Z)$ be a fixed matrix that satisfies $\mathcal{E}_{2n+1,p_i}^{\alt}$ for $1\le i\le l$, and $\Cok_{\tors}(A_{2n+1})_P\simeq G^{(1)},\corank(A_{2n+1})=1$. Then, we have
\begin{multline}\label{eq: odd to even isomorphic to G2_alt}
\mathbf{P}\left(\Cok\left(
\begin{array}{c|c}
A_{2n+1} & 
\begin{matrix}
\xi_1 \\
\vdots \\
\xi_{2n+1}
\end{matrix}
\\ \hline
\begin{matrix}
-\xi_1 & \cdots & -\xi_{2n+1}
\end{matrix}
& 0
\end{array}
\right)_P\simeq G^{(2)}\right)\\
=\prod_{i=1}^l\mathbf{P}\left(G_{p_i}^{(2)},\text{ even }\bigg| G_{p_i}^{(1)},\text{ odd}\right)+O_{G^{(1)},G^{(2)},P,\epsilon}(\exp(-\epsilon\Omega_{G^{(1)},G^{(2)},P}(n))).
\end{multline}
\end{enumerate}
\end{cor}

\begin{proof}
We will only prove the ``from even to odd'' case, since the ``from odd to even'' can be deduced routinely. Let $a:=p_1^{e_{p_1}}\ldots p_l^{e_{p_l}}$, where $e_{p_i}=\max\{\Dep_{p_i}(G^{(1)}),\Dep_{p_i}(G^{(2)})\}+1$ for all $1\le i\le l$. Let $\xi_1',\xi_2',\ldots$ be independent and uniformly distributed in $\{0,1,\ldots,a-1\}$. Denote by $\bm{\xi}:=(\xi_1,\ldots,\xi_{2n})$, and $\bm{\xi}':=(\xi_1',\ldots,\xi_{2n}')$. Applying \Cref{prop: universal transition_alt}, we have 
\begin{align}
\begin{split}
\text{LHS}\eqref{eq: even to odd isomorphic to G2_alt}&=\mathbf{P}\left(\Cok\begin{pmatrix}A_{2n}/a & \bm{\xi}'/a\\
-(\bm{\xi}'/a)^T & 0
\end{pmatrix}
\simeq G^{(2)}\oplus(\Z/a\Z)\right)\\
&=\mathbf{P}\left(\Cok\begin{pmatrix}A_{2n}/a & \bm{\xi}/a\\
-(\bm{\xi}/a)^T & 0
\end{pmatrix}
\simeq G^{(2)}\oplus(\Z/a\Z)\right)+O_{a,\epsilon}(\exp(-\epsilon\Omega_a(n)))\\
&=\prod_{i=1}^l\mathbf{P}\left(\Cok\begin{pmatrix}A_{2n}/p_i^{e_{p_i}} & \bm{\xi}'/p_i^{e_{p_i}}\\
-(\bm{\xi}'/p_i^{e_{p_i}})^T & 0
\end{pmatrix}
\simeq G^{(2)}\oplus(\Z/a\Z)\right)+O_{a,\epsilon}(\exp(-\epsilon\Omega_a(n)))\\
&=\prod_{i=1}^l\mathbf{P}\left(G_{p_i}^{(2)},\text{ odd }\bigg| G_{p_i}^{(1)},\text{ even}\right)+O_{a,\epsilon}(\exp(-\epsilon\Omega_a(n))).\\
&=\prod_{i=1}^l\mathbf{P}\left(G_{p_i}^{(2)},\text{ odd }\bigg| G_{p_i}^{(1)},\text{ even}\right)+O_{G^{(1)},G^{(2)},P,\epsilon}(\exp(-\epsilon\Omega_{G^{(1)},G^{(2)},P}(n))).
\end{split}
\end{align}
Here, the last line is due to the fact that $a$ depends on $G^{(1)},G^{(2)},P$.
\end{proof}

\begin{proof}[Proof of \Cref{thm: quotient simulate_alt}]
We will only prove the even size case, since the odd size case can be deduced routinely. Suppose $a$ has prime factorization $a=p_1^{e_{p_1}}\cdots p_l^{e_{p_l}}$. Consider the exposure process 
$$\Cok(A_{n/20}/a),\ldots,\Cok(A_{2n}/a)$$ 
obtained by iteratively revealing $A_t$ for $n/20\le t\le 2n$ by adding a new row and column each time and considering the resulting cokernel with quasi-pairing $\Cok(A_t/a)$. Also, we construct another exposure process 
$$\Cok(A_{n/20}'/a),\ldots,\Cok(A_{2n}'/a),$$
where $A_{n/20}':=A_{n/20}$, and the matrices $A_{n/20+1}',\ldots,A_{2n}'$ have entries in $\Z$ and are obtained by adding a uniformly distributed row and column in $\{0,1,\ldots,a-1\}$ every time.

As we will see, the random variable $\Cok(A_{2n}'/a)$ plays an intermediate role between $\mu_{2\infty}^{\alt,(a)}$ and the law of $\Cok(A_{2n}/a)$: that is, we first bound the $L^1$-distance between the law of $\Cok(A_{2n}/a)$ and $\Cok(A_{2n}'/a)$, and then bound the $L^1$-distance between $\mu_{2\infty}^{\alt,(a)}$ and the law of $\Cok(A_{2n}'/a)$.

\noindent{\bf Step 1. The $L^1$-distance between the law of $\Cok(A_{2n}/a)$ and $\Cok(A_{2n}'/a)$. }
We claim that for all $t\in[n/20,2n-1]$, we have
\begin{equation}\label{eq: increment of distance adding new row and column_alt}
D_{L_1}\left(\Cok(A_{t+1}/a),\Cok(A_{t+1}'/a)\right)-D_{L_1}\left(\Cok(A_t/a),\Cok(A_t'/a)\right)\\
=O_{a,\epsilon}(\exp(-\epsilon\Omega_{a}(t))).
\end{equation}
Indeed, by \Cref{thm: not sparse}, the matrix $A_t\in\Alt_t(\Z)$ satisfies the properties $\mathcal{E}_{t,p_1}^{\alt},\ldots,\mathcal{E}_{t,p_l}^{\alt}$ with probability at least
$$1-\sum_{i=1}^lO_{p_i}(\exp(-\epsilon\Omega_{p_i}(t)))=1-O_{a,\epsilon}(\exp(-\epsilon\Omega_{a}(t))).$$ 
As long as $A_t$ satisfies these properties, we can apply \Cref{prop: universal transition_alt} so that \eqref{eq: increment of distance adding new row and column_alt} immediately follows. Since $A_{n/20}'/a=A_{n/20}/a$, we deduce that
$$D_{L_1}\left(\Cok(A_{2n}/a),\Cok(A_{2n}'/a)\right)=\sum_{t=n/20}^{2n} O_{a,\epsilon}(\exp(-\epsilon\Omega_a(t)))=O_{a,\epsilon}(\exp(-\epsilon\Omega_a(n))).$$

\noindent{\bf Step 2. The $L^1$-distance between $\mu_{2\infty}^{\alt,(a)}$ and the law of $\Cok(A_{2n}'/a)$.} 
By \Cref{thm: not sparse}, the matrix $A_{n/20}'\in\Alt_{n/20}(\Z)$ satisfies the properties $\mathcal{E}_{n/20,p_1}^{\alt},\ldots,\mathcal{E}_{n/20,p_l}^{\alt}$ with probability at least
$$1-\sum_{i=1}^lO_{p_i}(\exp(-\epsilon\Omega_{p_i}(n/20)))=1-O_{a,\epsilon}(\exp(-\epsilon\Omega_a(n))).$$
As long as $A_{n/20}'$ satisfies these properties, we have
$\corank(A_{n/20}'/p_i)\le (n/20)^{2/3}\le n^{2/3}$ for all $1\le i\le l$. Moreover, the exposure process naturally factorizes over the primes $p_1,\ldots,p_l$ when we add new uniform rows and columns, since each newly revealed entry is uniform in $\mathbb Z/a\mathbb Z$, hence (by the Chinese remainder theorem) its reductions modulo $p_i^{e_{p_i}}$ are independent and uniform for $1\le i\le l$. Thus, conditioned on $A_{n/20}\in\Alt_{n/20}(\Z)$, the random variables
$$\Cok(A_{2n}'/p_i^{e_{p_i}}),\quad 1\le i\le l$$
are independent. Also, we have that for all $t\in[n/20,n-1]$ and $1\le i\le l$,
\begin{align}
\begin{split}
\mathbf{P}(\corank(A_{t+1}'/p_i)-\corank(A_t'/p_i)=-1\mid\corank(A_t'/p_i)\ge 2)&\ge 1-p_i^{-2},\\
\mathbf{P}(\corank(A_{t+1}'/p_i)-\corank(A_t'/p_i)=-1\mid\corank(A_t'/p_i)=1)&=1-p_i^{-1}.\\
\end{split}
\end{align}
Indeed, working modulo $p_i$, we choose an
invertible principal minor of maximal size, whose existence is guaranteed by
\Cref{lem: full rank principal minor}. Using paired row-column operations (i.e., congruence
transformations), we can eliminate the entries in the corresponding rows and columns outside this
minor, reducing to a block form with an invertible block and a remaining complement block of
size equal to the corank, from which the desired one-step transition estimate follows.

Therefore, when $\corank(A_{n/20}'/p_i)\le n^{2/3}$, we can apply a standard Hoeffding
large-deviation bound to deduce that with probability at least $1-O_{p_i}(\exp(-\Omega_
{p_i}(n)))$, there exists an even integer $\tau_i\in[n/20+1,n/2]$ such that $\corank(A_{\tau_i}'/p_i)=0$. Notice that for any fixed even integer $\tau_i\in[n/20+1,n/2]$, when we condition on $\corank(A_{\tau_i}'/p_i)=0$, we have
$$\Cok(A_{2n}'/p_i^{e_{p_i}})\stackrel{d}{=}\Cok(H_{2n-\tau_i}/p_i^{e_{p_i}}),$$
where $H_{2n-\tau_i}/p_i^{e_{p_i}}\in\Alt_{2n-\tau_i}(\Z/p_i^{e_{p_i}}\Z)$ is uniformly distributed. Therefore, recalling the exponential convergence given in \Cref{prop: uniform model exponential convergence_alt}, we have
\begin{equation}\label{eq: An' and H_infty distance at p_i}
D_{L^1}\left(\mathcal{L}\left(\Cok(A_{2n}'/p_i^{e_{p_i}})\Bigg|\mathcal{E}_{n/20,p_i}^{\alt}\right),\mu_{2\infty}^{\alt,(p_i^{e_{p_i}})}\right)=O_{p_i^{e_{p_i}}}(\exp(-\Omega_{p_i^{e_{p_i}}}(n))).
\end{equation}
Summing up the $O_{a,\epsilon}(\exp(-\epsilon\Omega_a(n)))$ probability lost when assuming $\mathcal{E}_{n/20,p_1}^{\alt},\ldots,\mathcal{E}_{n/20,p_l}^{\alt}$ for $A_{n/20}$, and the $L^1$-distance estimate \eqref{eq: An' and H_infty distance at p_i} for all $1\le i\le l$, we deduce that
$$D_{L_1}(\mathcal{L}(\Cok(A_{2n}'/a)),\mu_{2\infty}^{\alt,(a)})=O_{a,\epsilon}(\exp(-\epsilon\Omega_a(n))).$$

These two steps together provide the proof.
\end{proof}

As a byproduct, we also obtain the convergence of the joint distribution of corners, which is given by the following proposition.

\begin{prop}\label{prop: joint distribution_alt}
Let $\epsilon>0$ and $A_{2n}$ be the same as in \Cref{thm: exponential convergence_alt}. Moreover, for all $1\le t\le 2n$, let $A_t$ be the $t\times t$ upper-left corner of $A_{2n}$. Let $j\ge 1$ be a fixed integer, $P=\{p_1,\ldots,p_l\}$ be a finite set of primes, and
$$G^{(0)},\ldots,G^{(2j)}\in\mathcal{S}_P.$$ 
Then we have 
\begin{align}
\begin{split}
&\mathbf{P}\Bigg(\Cok(A_{2n-2j})_P\simeq G^{(0)},\Cok_{\tors}(A_{2n-2j+1})_P\simeq G^{(1)},\corank(A_{2n-2j+1})=1,\cdots,\\
&\Cok(A_{2n-2})_P\simeq G^{(2j-2)},\Cok_{\tors}(A_{2n-1})_P\simeq G^{(2j-1)},\corank(A_{2n-1})=1,\Cok(A_{2n})_P\simeq G^{(2j)}\Bigg)\\
&=\prod_{i_1=1}^l\Bigg(\frac{\prod_{k\ge 1}(1-p_{i_1}^{-1-2k})}{\left|\Sp\left(G_{p_{i_1}}^{(0)}\right)\right|}\cdot\prod_{i_2=0}^{j-1}\mathbf{P}\left(G^{(2i_2+2)}_{p_{i_1}},\text{ even }\bigg|G^{(2i_2+1)}_{p_{i_1}},\text{ odd}\right)\\
&\cdot\mathbf{P}\left(G^{(2i_2+1)}_{p_{i_1}},\text{ odd }\bigg|G^{(2i_2)}_{p_{i_1}},\text{ even}\right)\Bigg)+O_{G^{(0)},\ldots,G^{(2j)},P,\epsilon}(\exp(-\epsilon\Omega_{G^{(0)},\ldots,G^{(2j)},P}(n))).
\end{split}
\end{align}
\end{prop}

\begin{proof}
It suffices to prove the stronger statement that for all $1\le j_0\le j$, 
\begin{align}\label{eq: joint distribution for j0_alt} 
\begin{split}
&\mathbf{P}\Bigg(\Cok(A_{2n-2j})_P\simeq G^{(0)},\Cok_{\tors}(A_{2n-2j+1})_P\simeq G^{(1)},\corank(A_{2n-2j+1})=1,\cdots,\\
&\Cok_{\tors}(A_{2n-2j+2j_0-1})_P\simeq G^{(2j_0-1)},\corank(A_{2n-2j+2j_0-1})=1,\Cok(A_{2n-2j+2j_0})_P\simeq G^{(2j_0)}\Bigg)\\
&=\prod_{i_1=1}^l\Bigg(\frac{\prod_{k\ge 1}(1-p_{i_1}^{-1-2k})}{\left|\Sp\left(G_{p_{i_1}}^{(0)}\right)\right|}\cdot\prod_{i_2=0}^{j_0-1}\left(G^{(2i_2+2)}_{p_{i_1}},\text{ even }\bigg|G^{(2i_2+1)}_{p_{i_1}},\text{ odd}\right)\\
&\cdot\mathbf{P}\left(G^{(2i_2+1)}_{p_{i_1}},\text{ odd }\bigg|G^{(2i_2)}_{p_{i_1}},\text{ even}\right)\Bigg)+O_{G^{(0)},\ldots,G^{(2j_0)},P,\epsilon}(\exp(-\epsilon\Omega_{G^{(0)},\ldots,G^{(2j_0)},P}(n))).
\end{split}
\end{align}
The proof proceeds by induction over $j_0$. The case $j_0=1$ can be directly deduced from \Cref{thm: exponential convergence_alt}. Now, suppose \eqref{eq: joint distribution for j0_alt} already holds for some $1\le j_0\le j-1$. By \Cref{thm: not sparse}, with probability no less than $1-O_{P,\epsilon}(\exp(-\epsilon\Omega_{P}(n)))$, the matrix $A_{2n-2j+2j_0}$ satisfies the properties
$\mathcal{E}_{2n-2j+2j_0,p_1}^{\alt},\ldots,\mathcal{E}_{2n-2j+2j_0,p_l}^{\alt}$, and the matrix $A_{2n-2j+2j_0+1}$ satisfies the properties
$\mathcal{E}_{2n-2j+2j_0+1,p_1}^{\alt},\ldots,\mathcal{E}_{2n-2j+2j_0+1,p_l}^{\alt}$. Thus, applying \Cref{cor: integral universal transition_alt} once for the transition from even to odd and once from odd to even, 
we have
\begin{align}
\begin{split}
&\mathbf{P}\Bigg(\Cok(A_{2n-2j})_P\simeq G^{(0)},\Cok_{\tors}(A_{2n-2j+1})_P\simeq G^{(1)},\corank(A_{2n-2j+1})=1,\cdots,\\
&\Cok_{\tors}(A_{2n-2j+2j_0+1})_P\simeq  G^{(2j_0+1)},\corank(A_{2n-2j+2j_0+1})=1,\Cok(A_{2n-2j+2j_0+2})_P\simeq  G^{(2j_0+2)},\\
&\text{$A_{2n-2j+2j_0}$ satisfies the properties
$\mathcal{E}_{2n-2j+2j_0,p_1}^{\alt},\ldots,\mathcal{E}_{2n-2j+2j_0,p_l}^{\alt}$},\\
&\text{and $A_{2n-2j+2j_0+1}$ satisfies the properties
$\mathcal{E}_{2n-2j+2j_0+1,p_1}^{\alt},\ldots,\mathcal{E}_{2n-2j+2j_0+1,p_l}^{\alt}$} \Bigg)\\
&=\prod_{i_1=1}^l\Bigg(\frac{\prod_{k\ge 1}(1-p_{i_1}^{-1-2k})}{\left|\Sp\left(G_{p_{i_1}}^{(0)}\right)\right|}\cdot\prod_{i_2=0}^{j_0}\mathbf{P}\left(G^{(2i_2+2)}_{p_{i_1}},\text{ even }\bigg|G^{(2i_2+1)}_{p_{i_1}},\text{ odd}\right)\\
&\cdot\mathbf{P}\left(G^{(2i_2+1)}_{p_{i_1}},\text{ odd }\bigg|G^{(2i_2)}_{p_{i_1}},\text{ even}\right)\Bigg)+O_{G^{(0)},\ldots,G^{(2j_0+2)},P,\epsilon}(\exp(-\epsilon\Omega_{G^{(0)},\ldots,G^{(2j_0+2)},P}(n))).
\end{split}
\end{align}
Therefore, the statement \eqref{eq: joint distribution for j0_alt} also holds for $j_0+1$. This completes the proof. 
\end{proof}

\begin{rmk}
\Cref{prop: joint distribution_alt} shows that the joint law of the corner data admits universal asymptotics.
In particular, fixing a single prime $p$ (i.e.\ taking $P=\{p\}$), as $n\to\infty$ we obtain the
same limiting distribution as in the Haar model over $\mathbb Z_p$.
This Haar setting can be viewed as the non-archimedean analogue of the aGUE corners process studied in \cite{shen2024non}:
the transition kernel appears in \cite[Theorem 1.3]{shen2024non}, the limiting law is given in~\cite[(1.7)]{shen2024non},
and the full joint distribution of corners is described in \cite[Theorem 1.2]{shen2024non}.
Taken together, these results identify the corner process as a Hall--Littlewood process.

It is also worth mentioning that, in the present paper, we use the forward exposure process obtained by adjoining one new row and column at each step,
whereas \cite{shen2024non} studies the same joint distribution from the reverse direction:
one first samples the cokernel of the largest matrix and then applies the backward transition obtained by deleting a row and column at each step.
\end{rmk}



