\section{Preliminaries}\label{sec: Preliminaries}

\subsection{Preparations from coding theory}

\begin{defi}
For a prime number $p$, denote by 
$$H_p(x):=x\log_p(p-1)-x\log_p x-(1-x)\log_p(1-x),\quad 0<x<1$$
the $p$-\emph{ary entropy function}. 
\end{defi}

\begin{defi}
For all integers $0\le t\le n$ and prime numbers $p$,  denote by $\Vol_p(n,t)$ the order of the Hamming ball in $\F_p^n$ of radius $t$ (i.e., the elements in $\F_p^n$ that have at most $t$ nonzero entries). Clearly we have
$$\Vol_p(n,t)=\sum_{i=0}^t\binom{n}{i}(p-1)^i.$$
\end{defi}

The following proposition is quoted from \cite[Proposition 3.3.1]{guruswami2012essential}, which shows that the $p$-ary entropy function describes the asymptotics of the Hamming ball.

\begin{prop}\label{prop: asymptotic of Hamming ball}
Let $p$ be a prime, and $0<x<1-1/p$ be a real number. When $n$ tends to infinity, we have
$$p^{(H_p(x)-o(1))n}\le \Vol_p(n,xn)\le p^{H_p(x)n}.$$
\end{prop}

% \begin{defi}\label{defi: distance}
% Let $V$ be a linear subspace of $\F_p^n$. We denote 
% $$\dist(V):=\min_{\mathbf{v}\in V,\mathbf{v}\ne 0}\wt(\mathbf{v})$$
% as the \emph{distance} of $V$.
% \end{defi}

% If $\dist(V)=d$, then it is clear that every nonzero element of $V$ has at least $d$ nonzero entries. The following lemma is obtained by the greedy construction that leads to the Gilbert–Varshamov lower bound, see \cite[Section 4.2]{guruswami2012essential}.

% \begin{lemma}\label{lem: Gilbert–Varshamov}
% Suppose the linear subspace $V\subset\F_p^n$ satisfies $\dist(V)\ge d$. Then there exists a linear subspace $W\supset V$ with $\dist(W)\ge d$, and
% $$\dim(W)\ge n(1-H_p(d/n)).$$
% \end{lemma}

\subsection{Partitions and finite abelian groups}

\begin{defi}
We denote by $\Y$ the set of \emph{partitions} $\lambda=(\lambda_1,\lambda_2,\ldots)$, which are (finite or infinite) sequences of nonnegative integers $\lambda_1\ge\lambda_2\ge\cdots$ that are eventually zero. We do not distinguish between two such sequences that differ only by a string of zeros at the end. The integers $\l_i>0$ are called the \emph{parts} of $\l$. We set $|\l| := \sum_{i\ge 1}\l_i,n(\lambda):=\sum_{i\ge 1} (i-1)\lambda_i$, and $m_k(\l):= \#\{i\mid \l_i = k\}$. We also write $\Len(\lambda)=\#\{i\mid \l_i >0\}$ as the \emph{length} of $\lambda$, which is the number of nonzero parts.
\end{defi}

In the coming sections, we usually write $P$ for a finite set of prime numbers, and $p$ a prime number. We also denote by $\Q_p$ the $p$-adic field, $\val(\cdot):\Q_p^\times\rightarrow\Z$ the valuation function, $\Z_p:=\{\alpha\in\Q_p^\times:\val(\alpha)\ge 0\}\cup\{0\}$ the $p$-adic integers, and $\F_p:=\Z/p\Z$ the finite field of size $p$. For a finite abelian group $G$ and a prime $p$, we write $G_p$ for the Sylow $p$-subgroup of $G$, so that
$$G=\bigoplus_{p\text{ prime}}G_p.$$
We will also write $G_P:=\bigoplus_{p\in P} G_p$ the $P$-primary subgroup of $G$. As a finite abelian $p$-group, $G_p$ is isomorphic to $\bigoplus_{i\ge 1}(\Z/p^{\l_i}\Z)$ for some $\l=(\l_1,\l_2\ldots)\in\Y$. In this case, we say $G$ has type $\l$ at $p$, and we denote by $\Dep_p(G):=\l_1$ the $p$-\emph{depth} of $G$, i.e., the smallest integer such that $p^{\Dep_p(G)}G_p=0$. We also denote by $\Len_p(G):=\Len(\l)=n$ the $p$-\emph{length} of $G$, i.e., the number of cyclic factors in $G_p$. In particular, the $p$-torsion subgroup $G_p$ is trivial if and only if $\Dep_p(G)=\Len_p(G)=0$.

\subsection{Auxiliary tools}

In this subsection, we introduce some useful lemmas that will be useful in later sections. We recall the following basic linear algebra lemma about full-rank principal minors. The symmetric part is essentially contained in \cite[Lemma 3.1]{ferber2023random}, and the alternating part follows analogously.

\begin{lemma}\label{lem: full rank principal minor}
Let $p$ be a prime number, and $n\ge 1$. 
\begin{enumerate}
\item (Symmetric case) Suppose $M_n\in\Sym_n(\F_p)$ has rank $r$. Then there is a $r\times r$ principal minor with determinant in $\F_p^\times$.
\item (Alternating case) Suppose $A_n\in\Alt_n(\F_p)$ has rank $2r$. Then there is a $2r\times 2r$ principal minor with determinant in $\F_p^\times$.
\end{enumerate}
\end{lemma}

\begin{rmk}\label{rem:principal-minor-lifts}
\Cref{lem: full rank principal minor} has natural analogues for matrices with entries in
$\Z_p$ or in the quotient ring $\Z/p^{e_p}\Z$ (for any $e_p\ge 1$). Indeed, given
$M_n\in \Sym_n(\Z_p)$ (or $M_n\in \Sym_n(\Z/p^{e_p}\Z)$), one may reduce modulo $p$ to obtain a matrix in
$\Sym_n(\F_p)$ and apply the lemma there to find a full-rank principal minor. The corresponding
principal minor of the original matrix then has determinant in $\Z_p^\times$ (respectively in
$(\Z/p^{e_p}\Z)^\times$). The same applies in the alternating case.
In later arguments we will routinely use this lifted version of the lemma for matrices over $\Z_p$
or $\Z/p^{e_p}\Z$.
\end{rmk}

\begin{defi}
We say that a random variable $\xi\in\Z/a\Z$ is \emph{not $\epsilon$-concentrated} if $\mathbf{P}(\xi\equiv r\mod p)\le1-\epsilon$ for all $p\mid a$ and $r\in \F_p$. We also say that a random integer $\xi\in\Z$ is \emph{not $\epsilon$-concentrated at} $a$ if $\xi+a\Z\in\Z/a\Z$ is not $\epsilon$-concentrated.
\end{defi}

\begin{lemma}\label{lem: expectation of not concentrated}
Let $z$ be random and not $\epsilon$-concentrated in $\Z/a\Z$. Then we have
$$\left|\E\left(\exp\left(\frac{2\pi\sqrt{-1}}{a}z\right)\right)\right|\le\exp(-\frac{\epsilon}{a^2}).$$
\end{lemma}

\begin{proof}
See \cite[Lemma 4.2]{wood2017distribution}.
\end{proof}





